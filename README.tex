\hypertarget{ux7b80ux6d01ux7684-markdown-ux7f16ux8f91ux5668-smark}{%
\section{简洁的 Markdown 编辑器
Smark}\label{ux7b80ux6d01ux7684-markdown-ux7f16ux8f91ux5668-smark}}

说白了只是为了自己方便使用,并没什么新奇的东西。我使用
\href{http://johnmacfarlane.net/pandoc/}{pandoc} 来转化
\texttt{markdown},但是我不想在修改文件时总是在编辑器、文字终端和浏览器间换来换去,因此我写了一个简单的编辑器,它在后台调用
\texttt{pandoc} 将当前编辑的 \texttt{markdown} 内容转化为
\texttt{HTML},而后将 \texttt{HTML} 在 \texttt{smark}
中的浏览器中显示出来,就是这么回事。Smark 依赖于
\href{http://johnmacfarlane.net/pandoc/}{\texttt{pandoc}}、\href{http://qt-project.org/}{\texttt{Qt\ 4.8}}
和
\href{http://www.mathjax.org}{\texttt{MathJax}},在此向上述软件包开发者们致敬。请注意继承于
pandoc 的发布协议,Smark 同样遵循
\href{http://www.gnu.org/copyleft/gpl.html}{GPL},如有任何疑问请联系
\url{elerao.ao@gmail.com},我将尽快做出回复。

主要特性:

\begin{itemize}
\tightlist
\item
  \texttt{Windows} / \texttt{Linux} 等主流系统跨平台支持;
\item
  完美支持 \texttt{LaTex} 数学公式、脚注、尾注等,支持使用本地
  \texttt{MathJax} 调用,不需要在线访问 \texttt{MathJax\ CDN};
\item
  用户可配置的 \texttt{Markdown} 语法高亮显示,美观整洁;
\item
  多种格式文件导出支持,可将当前 \texttt{Markdown} 文件另存为
  \texttt{HTML}、
  \texttt{Miscrosoft\ Word}、\texttt{OpenOffice\ /\ LibreOffice\ ODT\ Document}、\texttt{Latex}、\texttt{PDF}、\texttt{reStructured\ Text}、\texttt{Media\ Wiki\ markup}、\texttt{epub}
  以及 \texttt{plain\ txt} 等格式文件输出;
\item
  可通过用户指明 \texttt{CSS} 文件定义显示式样;
\item
  简洁友好的界面布局,尽可能地凸显正在编辑的内容;
\item
  系统、完备的各类快捷键,极大地提高了工作效率;
\end{itemize}

\hypertarget{ux5b89ux88c5-smark}{%
\subsection{安装 Smark}\label{ux5b89ux88c5-smark}}

对于 \texttt{Win32} 平台的用户,您可以直接下载当前版本的可执行程序
\href{http://pan.baidu.com/s/1ntMCVFV}{Smark-X.X-Win32-portable.zip}。对于其它平台的用户,可以下载当前版本的
Smark 源代码
\href{http://pan.baidu.com/s/1ntMCVFV}{Smark-X.X-src.zip}在本平台上进行编译即可,一般地您可以按照如下步骤编译
\texttt{Smark}:

\begin{enumerate}
\def\labelenumi{\arabic{enumi}.}
\item
  安装依赖的程序包:

\begin{verbatim}
$ sudo apt-get install qtsdk
$ sudo apt-get install pandoc
\end{verbatim}

  如果您不使用 \texttt{apt-get}
  作为软件包管理器,自己谷歌搜索如何安装这两个程序包
\item
  下载 Smark 源代码
  \href{http://pan.baidu.com/s/1ntMCVFV}{Smark-X.X-src.zip},解压并编译:

\begin{verbatim}
$ qmake -project
$ qmake
$ make
\end{verbatim}
\item
  把编译出的可执行文件移动到您的 \texttt{{[}bin{]}} 路径下即可使用,此时
  \texttt{Smark} 将使用默认的 \texttt{CSS} 样式表和基于
  \texttt{MathJax\ CDN} 的 \texttt{MathJax} 访问。如果想要得到与
  \texttt{Windows} 平台下完全一致的体验,您还需下载
  \href{http://pan.baidu.com/s/1ntMCVFV}{smark-2.0-resource.zip} 解压倒
  \texttt{Smark} 的安装目录下并进行设置。
\end{enumerate}

\hypertarget{ux8fd0ux884cux622aux56fe}{%
\subsection{运行截图}\label{ux8fd0ux884cux622aux56fe}}

 Windows 7 下的 Smark 运行截图 Windows 7 下的 Smark 配置选项对话框

\hypertarget{faq}{%
\subsection{FAQ}\label{faq}}

\hypertarget{ux4f4d-windows-ux4e0bux63d0ux793aux65e0ux6cd5ux8fd0ux884c-pandoc}{%
\subsubsection{64 位 Windows 下提示无法运行
pandoc:}\label{ux4f4d-windows-ux4e0bux63d0ux793aux65e0ux6cd5ux8fd0ux884c-pandoc}}

这是因为
\href{http://pan.baidu.com/s/1ntMCVFV}{Smark-X.X-Win32-portable.zip}
中自带的 pandoc.exe 是 32 位系统下的,您可以通过下载安装包
\href{http://pan.baidu.com/s/1ntMCVFV\#path=\%252Fsmark}{pandoc-1.13.1-win64.msi}
直接安装 64 位的 \texttt{pandoc} 而删除
\href{http://pan.baidu.com/s/1ntMCVFV}{Smark-X.X-Win32-portable.zip}
中自带的 32 位的 \texttt{pandoc.exe}。

\hypertarget{ux8c37ux6b4cux8f93ux5165ux6cd5ux65e0ux6cd5ux8f93ux5165ux95eeux9898}{%
\subsubsection{谷歌输入法无法输入问题}\label{ux8c37ux6b4cux8f93ux5165ux6cd5ux65e0ux6cd5ux8f93ux5165ux95eeux9898}}

貌似大凡使用 \texttt{QTextEdit}
部件的地方都存在这样的问题,这是因为您未开启
谷歌输入法的内嵌编辑模式,可通过点击 ``谷歌输入法设置对话框'' 中
``设置内嵌编辑模式'' 按钮,后勾选 ``使用内嵌编辑模式''
即可,如下图所示:

\hypertarget{todo}{%
\subsection{TODO}\label{todo}}

\hypertarget{ux73b0ux6709ux95eeux9898}{%
\subsubsection{现有问题:}\label{ux73b0ux6709ux95eeux9898}}

\begin{itemize}
\tightlist
\item
  在 Qt5 下编译的 WebView 无法完整地加载 CSS 中的设置,Qt4 下没问题;
\item
  改动后的 Markdwon 语法高亮的超链接 和 加粗 的正则表达式匹配有问题;
\item
  导出 PDF 和打印时的内容分页问题;
\end{itemize}

\hypertarget{ux5f85ux6dfbux52a0ux529fux80fd}{%
\subsubsection{待添加功能}\label{ux5f85ux6dfbux52a0ux529fux80fd}}

\begin{itemize}
\tightlist
\item
  仿照 QtCreator 3.1.xx 中编辑器的多行同时编辑实现,支持 Sublime
  等软件所支持的多行同时编辑;
\item
  与 百度云、Google Drive、OneDrive 内容同步,这个以后再说;
\end{itemize}

\hypertarget{ux9644ux5f55smark-ux7684ux5febux6377ux952eux5217ux8868}{%
\subsection{附录:Smark
的快捷键列表}\label{ux9644ux5f55smark-ux7684ux5febux6377ux952eux5217ux8868}}

全局

\begin{verbatim}
Esc              : 逐步隐藏所有不必要的部件,退出全屏显示
Tab              : 增加所选诸行的缩进(四个空格)
Ctrl + Tab       : 减小所选诸行的缩进(四个空格)
\end{verbatim}

文件菜单

\begin{verbatim}
Ctrl + N         : 新建 markdown 文件
Ctrl + O         : 打开 markdown 文件
Ctrl + S         : 保存当前 markdown 文件
Ctrl + Shift + S : 将当前文件另存为支持的格式
Ctrl + W         : 关闭当前 markdown 文件
Ctrl + P         : 打印当前 markdown 文件
Ctrl + Q         : 退出 Smark 
\end{verbatim}

视图菜单

\begin{verbatim}
F6               : 预览模式视图
F7               : 阅读模式视图
F8               : 编辑模式视图
F11              : 进入 / 退出全屏显示
\end{verbatim}

编辑菜单

\begin{verbatim}
F5               : 刷新 HTML 显示
Ctrl + Shift + C : 查看 HTML 源代码
Ctrl + C         : 复制
Ctrl + X         : 剪切
Ctrl + P         : 粘贴
Ctrl + Z         : 撤消
Ctrl + Y         : 重做
Ctrl + F         : 查找
\end{verbatim}

插入菜单

\begin{verbatim}
Ctrl + Shift + P : 插入图片
Ctrl + Shift + L : 插入链接
Ctrl + Shift + M : 插入数学公式
\end{verbatim}

格式菜单

\begin{verbatim}
Ctrl + B         : 加粗
Ctrl + I         : 倾斜
Ctrl + U         : 下划线
Ctrl + ]         : 加大字号
Ctrl + [         : 减小字号
Ctrl + Down      : 下标
Ctrl + Up        : 上标
Ctrl + `         : 代码
Ctrl + '         : 引用
Ctrl + L         : 内容左对齐
Ctrl + R         : 内容右对齐
Ctrl + E         : 内容居中
\end{verbatim}
